\documentclass[10pt,landscape,a4paper]{CheatSheet}
\usepackage{multicol}
\usepackage{calc}
\usepackage{listings}
\usepackage{xtab}
\usepackage{csquotes}

\lstset{
  breaklines = true,
  postbreak=\raisebox{0ex}[0ex][0ex]{\ensuremath{\hookrightarrow\space}},
}


\setlength{\parindent}{0pt}
\setlength{\parskip}{0pt plus 0.5ex}

\title{Universal Cheat-Sheet} 
\author{A. C. Hinrichs\\\texttt{adrian.hinrichs@rwth-aachen.de}}
\date{\today}
\begin{document}
\raggedright
\footnotesize
\begin{multicols}{4}
\setlength{\premulticols}{1pt}
\setlength{\postmulticols}{1pt}
\setlength{\multicolsep}{1pt}
\setlength{\columnsep}{2pt}
\maketitle
\section{\LaTeX}
\subsubsection{\LaTeX{}mk}
\begin{lstlisting}[language=bash]
\$ latexmk -pvc -pdf -interaction=nonstopmode \$TEX-FILE
\end{lstlisting}

Wenn nur eine \texttt{.tex}-Datei im Verzeichniss vorhanden ist, kann \texttt{\$TEX-FILE}%$
weggelassen werden\\
\subsection{lstlistings}
Inlinelistings with
\begin{lstlisting}[language=tex]
\lstinline{<Code>}
\end{lstlisting}
\subsection{Other commands}
\enquote{Splitted} Footnotes (for use in eg. Math-Environments)
\begin{lstlisting}[language=tex]
  \footnotemark % Prints the mark
  
  \footnotetext{#1} %Appends the text to the footnote
\end{lstlisting}

\section{EMACS--Commands}
\begin{xtabular}{rl}
  Open File (new or existing) & C-x C-f\\
  Save file & C-x C-s\\
  Save file as & C-x C-w\\
  Toogle Read-Only-Mode &C-x C-q\\
  \\
  Change Encoding & C-x RET r \texttt{<ENCODING>}\\
\end{xtabular}
\section{Jekyll}
\subsection{Jekyll starten}
\begin{lstlisting}[language=bash]
\$ jekyll serve
\end{lstlisting}
oder
\begin{lstlisting}[language=bash]
\$ bundle exec jekyll serve
\end{lstlisting}


\section{Markdown}
\subsection{Headers}
\begin{lstlisting}
  # H1
  ## H2
  ### H3
  #### H4
  ##### H5
  ###### H6
\end{lstlisting}  
 Alternatively, for H1 and H2, an underline-ish style:
\begin{lstlisting}
  Alt-H1
  ======
  
  Alt-H2
  ------
\end{lstlisting}
\subsection{Code}
Inline \lstinline{`code`} has \lstinline{`back-ticks around`} it.\\
Code-Blocks:
\begin{lstlisting}
  ```python
  s = "Python syntax highlighting"
  print s
  ```
  
  ```
  No language indicated, so no syntax highlighting. 
  But let's throw in a <b>tag</b>.
  ```
\end{lstlisting}
\subsection{Links}
\begin{lstlisting}
  [Display Text]({{ site.baseurl }}/pfad/zur/datei.html)
\end{lstlisting}
\end{multicols}
\end{document}