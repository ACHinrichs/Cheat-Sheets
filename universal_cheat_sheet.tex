\documentclass[10pt,landscape,a4paper]{article}
\usepackage{multicol}
\usepackage{calc}
\usepackage{ifthen}
\usepackage[landscape]{geometry}
\usepackage{amsmath,amsthm,amsfonts,amssymb}
\usepackage{color,graphicx,overpic}
\usepackage{hyperref}
\usepackage{listings}
\usepackage{xtab}
\usepackage{csquotes}

\lstset{
  breaklines = true,
  postbreak=\raisebox{0ex}[0ex][0ex]{\ensuremath{\hookrightarrow\space}},
}


\pdfinfo{
  /Title (universal_cheat_sheet.pdf)
  /Creator (TeX)
  /Producer (pdfTeX 1.40.0)
  /Author (ACHinrichs)
  /Subject (Universal Cheat Sheet)
  /Keywords (pdflatex, latex,pdftex,tex)}

% This sets page margins to .5 inch if using letter paper, and to 1cm
% if using A4 paper. (This probably isn't strictly necessary.)
% If using another size paper, use default 1cm margins.
\ifthenelse{\lengthtest { \paperwidth = 11in}}
    { \geometry{top=.5in,left=.5in,right=.5in,bottom=.5in} }
    {\ifthenelse{ \lengthtest{ \paperwidth = 297mm}}
        {\geometry{top=1cm,left=1cm,right=1cm,bottom=1cm} }
        {\geometry{top=1cm,left=1cm,right=1cm,bottom=1cm} }
    }

% Turn off header and footer
\pagestyle{empty}

% Redefine section commands to use less space
\makeatletter
\renewcommand{\section}{\hrule\@startsection{section}{1}{0mm}%
                                {-1ex plus -.5ex minus -.2ex}%
                                {0.5ex plus .2ex}%x
                                {\normalfont\large\bfseries}}
\renewcommand{\subsection}{\@startsection{subsection}{2}{0mm}%
                                {-1explus -.5ex minus -.2ex}%
                                {0.5ex plus .2ex}%
                                {\normalfont\normalsize\bfseries}}
\renewcommand{\subsubsection}{\@startsection{subsubsection}{3}{0mm}%
                                {-1ex plus -.5ex minus -.2ex}%
                                {1ex plus .2ex}%
                                {\normalfont\small\bfseries}}
\makeatother

% Don't print section numbers
\setcounter{secnumdepth}{0}


\setlength{\parindent}{0pt}
\setlength{\parskip}{0pt plus 0.5ex}

%My Environments
\newtheorem{example}[section]{Example}
% -----------------------------------------------------------------------

\begin{document}
\raggedright
\footnotesize
\begin{multicols}{3}


% multicol parameters
% These lengths are set only within the two main columns
%\setlength{\columnseprule}{0.25pt}
\setlength{\premulticols}{1pt}
\setlength{\postmulticols}{1pt}
\setlength{\multicolsep}{1pt}
\setlength{\columnsep}{2pt}

\begin{center}
     \Large{\underline{Universal Cheat-Sheet}} \\
\end{center}

\section{\LaTeX}
\subsubsection{\LaTeX{}mk}
\begin{lstlisting}[language=bash]
\$ latexmk -pvc -pdf -interaction=nonstopmode \$TEX-FILE
\end{lstlisting}

Wenn nur eine \texttt{.tex}-Datei im Verzeichniss vorhanden ist, kann \texttt{\$TEX-FILE}%$
weggelassen werden\\
\subsection{lstlistings}
Inlinelistings with
\begin{lstlisting}[language=tex]
\lstinline{<Code>}
\end{lstlisting}
\subsection{Other commands}
\enquote{Splitted} Footnotes (for use in eg. Math-Environments)
\begin{lstlisting}[language=tex]
  \footnotemark % Prints the mark
  
  \footnotetext{#1} %Appends the text to the footnote
\end{lstlisting}

\section{EMACS--Commands}
\begin{xtabular}{rl}
  Open File (new or existing) & C-x C-f\\
  Save file & C-x C-s\\
  Save file as & C-x C-w\\
  Toogle Read-Only-Mode &C-x C-q\\
  \\
  Change Encoding & C-x RET r \texttt{<ENCODING>}\\
\end{xtabular}
\section{Jekyll}
\subsection{Jekyll starten}
\begin{lstlisting}[language=bash]
\$ jekyll serve
\end{lstlisting}
oder
\begin{lstlisting}[language=bash]
\$ bundle exec jekyll serve
\end{lstlisting}


\section{Markdown}
\subsection{Headers}
\begin{lstlisting}
  # H1
  ## H2
  ### H3
  #### H4
  ##### H5
  ###### H6
  
  Alternatively, for H1 and H2, an underline-ish style:
  
  Alt-H1
  ======
  
  Alt-H2
  ------
\end{lstlisting}
\subsection{Code}
Inline:
\begin{lstlisting}
  Inline `code` has `back-ticks around` it.
\end{lstlisting}
Code-Blocks:
\begin{lstlisting}
  ```javascript
  var s = "JavaScript syntax highlighting";
  alert(s);
  ```
  
  ```python
  s = "Python syntax highlighting"
  print s
  ```
  
  ```
  No language indicated, so no syntax highlighting. 
  But let's throw in a <b>tag</b>.
  ```
\end{lstlisting}
\subsection{Links}
\begin{lstlisting}
  [Display Text]({{ site.baseurl }}/pfad/zur/datei.html)
\end{lstlisting}
\end{multicols}
\end{document}