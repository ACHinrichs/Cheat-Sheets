\documentclass[10pt,landscape,a4paper]{cheatsheet}
\usepackage[ngerman]{babel}
\usepackage{float}
\usepackage{xtab}
\title{Universal Cheat-Sheet} 
\author{A. C. Hinrichs\\\texttt{adrian.hinrichs@rwth-aachen.de}}
\date{\today}
\begin{document}
  \maketitle
  \section{\LaTeX--Tools}
  \subsection{\LaTeX{}mk}
  \begin{lstlisting}[language=bash]
    $ latexmk -pvc -pdf -interaction=nonstopmode $TEX-FILE
  \end{lstlisting} % $ %Hack for Math-highlighting
  
  Wenn nur eine \texttt{.tex}-Datei im Verzeichniss vorhanden ist, kann \texttt{\$TEX-FILE}%$
  weggelassen werden\\
  \subsection{tlmgr}
  Um Pakete unter TexLive zu installieren
  \begin{lstlisting}
     # tlmgr install $PACKET
  \end{lstlisting} % $ %Hack for Math-highlighting
  \section{\LaTeX--Syntax}
  \subsection{lstlistings}
  Inlinelistings with
  \begin{lstlisting}[language=tex]
    \lstinline{<Code>}
  \end{lstlisting}
  \subsection{Other commands}
  \enquote{Splitted} Footnotes (for use in eg. Math-Environments)
  \begin{lstlisting}[language=tex]
    \footnotemark % Prints the mark
    
    \footnotetext{#1} %Appends the text to the footnote
  \end{lstlisting}
   
  \section{EMACS}
  \subsection{File-Management}
  \begin{tabular}{rl}
    Open File (new or existing) & C-x C-f\\
    Save file & C-x C-s\\
    Save file as & C-x C-w\\
  \end{tabular}
  \subsection{Buffers}
  \begin{tabular}{rl}
    Move to Buffer...&\\
    ...with name     & C-x b \texttt{<Name>}\\
    ...next          & C-x \(\rightarrow\)\\
    ...previous      & C-x \(\leftarrow\)\\
    \\
    Kill Buffer \texttt{<X>} & C-x k \texttt{<X>}\\
    \\
    Toogle Read-Only-Mode &C-x C-q\\
    Change Encoding  & C-x RET r \texttt{<ENC>}\
  \end{tabular}
  \subsection{I-Search}
  \begin{tabular}{rl}
    Search forward  & C-s \texttt{<String>}\\
    Search backward & C-r \texttt{<String>}\\
    Continue Search & C-s C-s \emph{or}\\
                    & C-r C-r\\
  \end{tabular}
  \subsection{Transpose}
  \begin{tabular}{rl}
    Transpose Chars  & C-t\\
    Transpose Words  & M-t\\
    Transpose Expressions C-M-t
  \end{tabular}
  Example: \(12\underline{3}45\overset{\textrm{C-t}}\longmapsto
  132\underline{4}5\), where the underline indicates the cursor
  position.
  \subsection{Kexboard-Macros}
  \begin{tabular}{rl}
    Begin recording & C-x (\\
    End recording   & C-x )\\
    Execute Macro   & C-x e\\
  \end{tabular}
  \subsection{Uncategorized}
  \begin{tabular}{rl}
    Repeat \textrm{<CMD>} \textrm{<N>} times & C-u \textrm{<N>} \textrm{<CMD>}\\
    Open second (e)shell & C-u M-x eshell\\
    Input Unicode Character & C-x 8 RET <ID> 
  \end{tabular}
  \section{EMACS Org-Mode}
  \subsection{Lists}
  \begin{tabular}{rl}
    Insert new item       & M-RET\\
    Insert new item \& Checkbox & M-S-RET\\
    De/Increase intendation & M-left/right\\
    De/Increase tree intedation & M-S-left/right\\
    Change bullet-style   & C-c -\\
    Listitem to Headline  & C-c *\\
    List to Headline-Subtree& C-c C-*\\
    Sort List             & C-c \^{}\\
    Toggle checkbox ToDo  & C-c C-c\\
  \end{tabular}
  \subsection{Uncategorized}
  \begin{tabular}{rl}
    Toggle ToDo-Tag & C-c C-t\\
    Export (see help-view)& C-c C-e <X> <Y>\\
  \end{tabular}
  \section{Jekyll}
  \subsection{Jekyll starten}
  \begin{lstlisting}[language=bash]
    $ jekyll serve
  \end{lstlisting} % $ %Hack for Math-highlighting
  oder
  \begin{lstlisting}[language=bash]
    $ bundle exec jekyll serve
  \end{lstlisting} % $ %Hack for Math-highlighting
  
  \section{Markdown}
  \subsection{Headers}
  \begin{lstlisting}
    # H1
    ## H2
    ### H3
    #### H4
    ##### H5
    ###### H6
  \end{lstlisting}  
  Alternatively, for H1 and H2, an underline-ish style:
  \begin{lstlisting}
    Alt-H1
    ======
    
    Alt-H2
    ------
  \end{lstlisting}
  \subsection{Code}
  Inline \lstinline{`code`} has \lstinline{`back-ticks around`} it.\\
  Code-Blocks:
  \begin{lstlisting}
    ```python
    s = "Python syntax highlighting"
    print s
    ```
    
    ```
    No language indicated, so no syntax highlighting. 
    But let's throw in a <b>tag</b>.
    ```
  \end{lstlisting}
  \subsection{Links}
  \begin{lstlisting}
    [Display Text]({{ site.baseurl }}/pfad/zur/datei.html)
  \end{lstlisting}
  \section{Screen}
  \subsection{Tastaturbefehle}
  Funktionierne in einer Sitzung:
  \begin{table}[H]
    \footnotesize
    \begin{xtabular}{rl}
      Ein neues Fenster �ffnen& C-a-c\\
      Zwischen Fenstern wechseln& C-a-SPC\\
      Sitzung Trennen & C-a-d\\
    \end{xtabular}
  \end{table}
  \subsection{Shell-Kommandos}
  Funktionieren ausserhalb von Sitzungen\\
  Zum start einer neuen Sitzung mit dem Namen \texttt{sitzung1}:
  \begin{lstlisting}
    $ screen -S sitzung1
  \end{lstlisting}%$
  Zu wiederaufnehmen der Sitzung \texttt{\$SITZUNG}:%$
  \begin{lstlisting}
    $ screen -r $SITZUNG
  \end{lstlisting}%$
  Wenn nur eine Sitzung im hintergrund l�uft kann der Sitzungsname weg
  gelassen werden.\\
  Zum auflisten aller Sitzungen:
  \begin{lstlisting}
    $ screen -ls 
  \end{lstlisting}%$
  Zum trennen einer (nicht ge�ffneten) Sitzung \texttt{\$SITZUNG}:
  \begin{lstlisting}
    $ screen -d $SITZUNG
  \end{lstlisting}
  Spiegeln der Sitzung \texttt{\$SITZUNG}:
  \begin{lstlisting}
    $ screen -rx $SITZUNG
  \end{lstlisting}
  Einen Befehl an die Sitzung \texttt{\$SITZUNG} senden und ausf�hren:
  \begin{lstlisting}
    $ screen -S $SITZUNG -X stuff $'ls -l\n'
  \end{lstlisting}%$
  \section{tar}
  Basic usage:
  \lstinline[language=Bash]{$ tar OPTIONS [-f OUTFILE] INFILES}% $ %Hack for Math-highlighting
  \subsection{Arguments}
  \begin{table}[H]
    \footnotesize
    \begin{xtabular}{rl}
      -c & \textbf{C}reate new Archive\\
      -u & \textbf{U}pdate Archive\\
      -r & Append to Archive\\
      -x & E\textbf{x}tract\\
    \end{xtabular}
  \end{table}
  \subsubsection{Output}
  \begin{table}[H]
    \footnotesize
    \begin{xtabular}{rl}
      -f & Specify Output\textbf{f}ile\\
      -O & Extract to std-\textbf{o}ut\\
    \end{xtabular}
  \end{table}
  \subsubsection{Compression}
  \begin{table}[H]
    \footnotesize
    \begin{xtabular}{rl}
      -j & Use bzip2\\
      -J & Use xz\\
      -z & Use gzip\\
      -Z & Use compress\\
    \end{xtabular}
  \end{table}
\end{document}